\documentclass[]{book}
\usepackage{lmodern}
\usepackage{amssymb,amsmath}
\usepackage{ifxetex,ifluatex}
\usepackage{fixltx2e} % provides \textsubscript
\ifnum 0\ifxetex 1\fi\ifluatex 1\fi=0 % if pdftex
  \usepackage[T1]{fontenc}
  \usepackage[utf8]{inputenc}
\else % if luatex or xelatex
  \ifxetex
    \usepackage{mathspec}
  \else
    \usepackage{fontspec}
  \fi
  \defaultfontfeatures{Ligatures=TeX,Scale=MatchLowercase}
\fi
% use upquote if available, for straight quotes in verbatim environments
\IfFileExists{upquote.sty}{\usepackage{upquote}}{}
% use microtype if available
\IfFileExists{microtype.sty}{%
\usepackage[]{microtype}
\UseMicrotypeSet[protrusion]{basicmath} % disable protrusion for tt fonts
}{}
\PassOptionsToPackage{hyphens}{url} % url is loaded by hyperref
\usepackage[unicode=true]{hyperref}
\hypersetup{
            pdftitle={A First Course In Statistics},
            pdfauthor={Andrea Mascaretti},
            pdfborder={0 0 0},
            breaklinks=true}
\urlstyle{same}  % don't use monospace font for urls
\usepackage{natbib}
\bibliographystyle{apalike}
\usepackage{color}
\usepackage{fancyvrb}
\newcommand{\VerbBar}{|}
\newcommand{\VERB}{\Verb[commandchars=\\\{\}]}
\DefineVerbatimEnvironment{Highlighting}{Verbatim}{commandchars=\\\{\}}
% Add ',fontsize=\small' for more characters per line
\usepackage{framed}
\definecolor{shadecolor}{RGB}{248,248,248}
\newenvironment{Shaded}{\begin{snugshade}}{\end{snugshade}}
\newcommand{\KeywordTok}[1]{\textcolor[rgb]{0.13,0.29,0.53}{\textbf{#1}}}
\newcommand{\DataTypeTok}[1]{\textcolor[rgb]{0.13,0.29,0.53}{#1}}
\newcommand{\DecValTok}[1]{\textcolor[rgb]{0.00,0.00,0.81}{#1}}
\newcommand{\BaseNTok}[1]{\textcolor[rgb]{0.00,0.00,0.81}{#1}}
\newcommand{\FloatTok}[1]{\textcolor[rgb]{0.00,0.00,0.81}{#1}}
\newcommand{\ConstantTok}[1]{\textcolor[rgb]{0.00,0.00,0.00}{#1}}
\newcommand{\CharTok}[1]{\textcolor[rgb]{0.31,0.60,0.02}{#1}}
\newcommand{\SpecialCharTok}[1]{\textcolor[rgb]{0.00,0.00,0.00}{#1}}
\newcommand{\StringTok}[1]{\textcolor[rgb]{0.31,0.60,0.02}{#1}}
\newcommand{\VerbatimStringTok}[1]{\textcolor[rgb]{0.31,0.60,0.02}{#1}}
\newcommand{\SpecialStringTok}[1]{\textcolor[rgb]{0.31,0.60,0.02}{#1}}
\newcommand{\ImportTok}[1]{#1}
\newcommand{\CommentTok}[1]{\textcolor[rgb]{0.56,0.35,0.01}{\textit{#1}}}
\newcommand{\DocumentationTok}[1]{\textcolor[rgb]{0.56,0.35,0.01}{\textbf{\textit{#1}}}}
\newcommand{\AnnotationTok}[1]{\textcolor[rgb]{0.56,0.35,0.01}{\textbf{\textit{#1}}}}
\newcommand{\CommentVarTok}[1]{\textcolor[rgb]{0.56,0.35,0.01}{\textbf{\textit{#1}}}}
\newcommand{\OtherTok}[1]{\textcolor[rgb]{0.56,0.35,0.01}{#1}}
\newcommand{\FunctionTok}[1]{\textcolor[rgb]{0.00,0.00,0.00}{#1}}
\newcommand{\VariableTok}[1]{\textcolor[rgb]{0.00,0.00,0.00}{#1}}
\newcommand{\ControlFlowTok}[1]{\textcolor[rgb]{0.13,0.29,0.53}{\textbf{#1}}}
\newcommand{\OperatorTok}[1]{\textcolor[rgb]{0.81,0.36,0.00}{\textbf{#1}}}
\newcommand{\BuiltInTok}[1]{#1}
\newcommand{\ExtensionTok}[1]{#1}
\newcommand{\PreprocessorTok}[1]{\textcolor[rgb]{0.56,0.35,0.01}{\textit{#1}}}
\newcommand{\AttributeTok}[1]{\textcolor[rgb]{0.77,0.63,0.00}{#1}}
\newcommand{\RegionMarkerTok}[1]{#1}
\newcommand{\InformationTok}[1]{\textcolor[rgb]{0.56,0.35,0.01}{\textbf{\textit{#1}}}}
\newcommand{\WarningTok}[1]{\textcolor[rgb]{0.56,0.35,0.01}{\textbf{\textit{#1}}}}
\newcommand{\AlertTok}[1]{\textcolor[rgb]{0.94,0.16,0.16}{#1}}
\newcommand{\ErrorTok}[1]{\textcolor[rgb]{0.64,0.00,0.00}{\textbf{#1}}}
\newcommand{\NormalTok}[1]{#1}
\usepackage{longtable,booktabs}
% Fix footnotes in tables (requires footnote package)
\IfFileExists{footnote.sty}{\usepackage{footnote}\makesavenoteenv{long table}}{}
\usepackage{graphicx,grffile}
\makeatletter
\def\maxwidth{\ifdim\Gin@nat@width>\linewidth\linewidth\else\Gin@nat@width\fi}
\def\maxheight{\ifdim\Gin@nat@height>\textheight\textheight\else\Gin@nat@height\fi}
\makeatother
% Scale images if necessary, so that they will not overflow the page
% margins by default, and it is still possible to overwrite the defaults
% using explicit options in \includegraphics[width, height, ...]{}
\setkeys{Gin}{width=\maxwidth,height=\maxheight,keepaspectratio}
\IfFileExists{parskip.sty}{%
\usepackage{parskip}
}{% else
\setlength{\parindent}{0pt}
\setlength{\parskip}{6pt plus 2pt minus 1pt}
}
\setlength{\emergencystretch}{3em}  % prevent overfull lines
\providecommand{\tightlist}{%
  \setlength{\itemsep}{0pt}\setlength{\parskip}{0pt}}
\setcounter{secnumdepth}{5}
% Redefines (sub)paragraphs to behave more like sections
\ifx\paragraph\undefined\else
\let\oldparagraph\paragraph
\renewcommand{\paragraph}[1]{\oldparagraph{#1}\mbox{}}
\fi
\ifx\subparagraph\undefined\else
\let\oldsubparagraph\subparagraph
\renewcommand{\subparagraph}[1]{\oldsubparagraph{#1}\mbox{}}
\fi

% set default figure placement to htbp
\makeatletter
\def\fps@figure{htbp}
\makeatother

\usepackage{booktabs}
\usepackage{amsthm}
\makeatletter
\def\thm@space@setup{%
  \thm@preskip=8pt plus 2pt minus 4pt
  \thm@postskip=\thm@preskip
}
\makeatother

\title{A First Course In Statistics}
\author{Andrea Mascaretti}
\date{2020-02-06}

\begin{document}
\maketitle

{
\setcounter{tocdepth}{1}
\tableofcontents
}
\chapter{Prerequisites}\label{prerequisites}

Welcome to this introductory course in statistics with R! This book will
help you get started and will guide you through the material of the
course.

The material of this course is based on the \textbf{moxier} package. It
consists of a series of \textbf{learnr} notebooks, \citep{R-learnr},
that will guide through your first steps into the wonderful world of
statistics and statistical computing!

\section{Getting started}\label{getting-started}

Whilst we will dedicate Chapter \ref{intro} to the topic of software
installation, we here introduce the main tools we will use throughout
this course: \textbf{R} and \textbf{RStudio}.

\subsection{R}\label{r}

So, what do we talk about when we talk about R? According to the
\href{https://www.r-project.org/about.html}{R project} website, R is a
software environment that includes:

\begin{quote}
\begin{itemize}
\tightlist
\item
  an effective data handling and storage facility,
\item
  a suite of operators for calculations on arrays, in particular
  matrices,
\item
  a large, coherent, integrated collection of intermediate tools for
  data analysis,
\item
  graphical facilities for data analysis and display either on-screen or
  on hardcopy, and
\item
  a well-developed, simple and effective programming language which
  includes conditionals, loops, user-defined recursive functions and
  input and output facilities.
\end{itemize}
\end{quote}

Quite a number of things! We will use this set of tools to dive deep
into the principles of statistics and statistical computation.

Another thing that is worth noticing is that \textbf{R} is Free
Software. It means anybody can contribute to its development. Common
tools have emerged to solve problems. \textbf{R} can be extended with
such tools, which are called \emph{packages} in R parlance, to do all
sort of incredible things. Packages are usually stored on CRAN, the
Comprehensive R Archive Network, and they range from packages to
\href{https://rich-iannone.github.io/blastula/}{send emails} to
\href{https://cran.r-project.org/web/views/MachineLearning.html}{machine
learning}.

\subsection{RStudio}\label{rstudio}

\href{https://rstudio.com/}{\textbf{RStudio}} is an \textbf{R} IDE
(Integrated Development Environment). But what does it mean? Basically,
it is a set of software that helps you be more productive: it allows you
to quickly manage files, see what variables you have defined and a vast
number of other things. This book and the \textbf{moxier} package have
been developed from within \textbf{RStudio}. We will see in a minute how
to install it!

\section{Some useful links}\label{some-useful-links}

If you are interesting in learning more about \textbf{R} as a
programming language, you can find many resources on the Internet. Some
nice books are \textbf{Hands-On Programming with R},
\citep{grolemund2014} to get started and \textbf{Advanced R}
\citep{wickham2019} to dive deep into the features of the language.

\section{Licence}\label{licence}

The \textbf{moxier} package is subject to the
\href{https://www.r-project.org/Licenses/GPL-3}{GPL-3} licence. For more
information, visit \url{https://mascaretti.github.io/moxier/}.

This book is licensed under a Creative Commons
Attribution-NonCommercial-ShareAlike 4.0 International License.

\chapter{Introduction}\label{intro}

We are now going to install the tools we need. We will start from
\textbf{R}, to then move to \textbf{RStudio} and, finally, the packages
we are going to need.

\section{Installing R}\label{installing-r}

Installing \textbf{R} has never been easier! You can simply follow the
instructions you find in the video.

install-R from RStudio, Inc. on Vimeo.

An alternative is Microsoft R Open, which can be downloaded
\href{https://mran.microsoft.com/}{here}. It lags behind the current
\textbf{R} version, but it comes with better performance tweaks
included.

\section{Installing RStudio}\label{installing-rstudio}

Once you have got \textbf{R} installing, simply follow this other video.

Install RStudio from RStudio, Inc. on Vimeo.

\section{Installing packages}\label{installing-packages}

We now cover the installation of packages. Packages extend the amount of
things you can do with \textbf{R}, from plotting to data analysis. We
will be using the \textbf{moxier} package through this course, for
instance.

To quickly get up to speed as to how install packages, again, simply
follow the video. Just, for the time being, skip the \textbf{tidyverse}
installation.

install Packages from RStudio, Inc. on Vimeo.

\section{\texorpdfstring{Installing
\textbf{moxier}}{Installing moxier}}\label{installing-moxier}

We are now ready to install \textbf{moxier}! First, we need to install
an auxiliary package, the \textbf{remotes} package. This package allows
to install other packages from GitHub. GitHub is a place where many
people develop and publish open-source software.

To install \textbf{remotes}, simply type in your console

\begin{Shaded}
\begin{Highlighting}[]
\KeywordTok{install.packages}\NormalTok{(}\StringTok{"remotes"}\NormalTok{)}
\end{Highlighting}
\end{Shaded}

Once the installation is complete, you can install \textbf{moxier}

\begin{Shaded}
\begin{Highlighting}[]
\NormalTok{remotes}\OperatorTok{::}\KeywordTok{install_github}\NormalTok{(}\StringTok{"mascaretti/moxier"}\NormalTok{)}
\end{Highlighting}
\end{Shaded}

Voilà! You are ready to start learning!

\chapter{Literature}\label{literature}

Here is a review of existing methods.

\chapter{Methods}\label{methods}

We describe our methods in this chapter.

\chapter{Applications}\label{applications}

Some \emph{significant} applications are demonstrated in this chapter.

\section{Example one}\label{example-one}

\section{Example two}\label{example-two}

\chapter{Final Words}\label{final-words}

We have finished a nice book.

\bibliography{book.bib,packages.bib}

\end{document}
