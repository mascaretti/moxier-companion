% Options for packages loaded elsewhere
\PassOptionsToPackage{unicode}{hyperref}
\PassOptionsToPackage{hyphens}{url}
%
\documentclass[
]{book}
\usepackage{lmodern}
\usepackage{amssymb,amsmath}
\usepackage{ifxetex,ifluatex}
\ifnum 0\ifxetex 1\fi\ifluatex 1\fi=0 % if pdftex
  \usepackage[T1]{fontenc}
  \usepackage[utf8]{inputenc}
  \usepackage{textcomp} % provide euro and other symbols
\else % if luatex or xetex
  \usepackage{unicode-math}
  \defaultfontfeatures{Scale=MatchLowercase}
  \defaultfontfeatures[\rmfamily]{Ligatures=TeX,Scale=1}
\fi
% Use upquote if available, for straight quotes in verbatim environments
\IfFileExists{upquote.sty}{\usepackage{upquote}}{}
\IfFileExists{microtype.sty}{% use microtype if available
  \usepackage[]{microtype}
  \UseMicrotypeSet[protrusion]{basicmath} % disable protrusion for tt fonts
}{}
\makeatletter
\@ifundefined{KOMAClassName}{% if non-KOMA class
  \IfFileExists{parskip.sty}{%
    \usepackage{parskip}
  }{% else
    \setlength{\parindent}{0pt}
    \setlength{\parskip}{6pt plus 2pt minus 1pt}}
}{% if KOMA class
  \KOMAoptions{parskip=half}}
\makeatother
\usepackage{xcolor}
\IfFileExists{xurl.sty}{\usepackage{xurl}}{} % add URL line breaks if available
\IfFileExists{bookmark.sty}{\usepackage{bookmark}}{\usepackage{hyperref}}
\hypersetup{
  pdftitle={A First Course In Statistics},
  pdfauthor={Andrea Mascaretti},
  hidelinks,
  pdfcreator={LaTeX via pandoc}}
\urlstyle{same} % disable monospaced font for URLs
\usepackage{longtable,booktabs}
% Correct order of tables after \paragraph or \subparagraph
\usepackage{etoolbox}
\makeatletter
\patchcmd\longtable{\par}{\if@noskipsec\mbox{}\fi\par}{}{}
\makeatother
% Allow footnotes in longtable head/foot
\IfFileExists{footnotehyper.sty}{\usepackage{footnotehyper}}{\usepackage{footnote}}
\makesavenoteenv{longtable}
\usepackage{graphicx}
\makeatletter
\def\maxwidth{\ifdim\Gin@nat@width>\linewidth\linewidth\else\Gin@nat@width\fi}
\def\maxheight{\ifdim\Gin@nat@height>\textheight\textheight\else\Gin@nat@height\fi}
\makeatother
% Scale images if necessary, so that they will not overflow the page
% margins by default, and it is still possible to overwrite the defaults
% using explicit options in \includegraphics[width, height, ...]{}
\setkeys{Gin}{width=\maxwidth,height=\maxheight,keepaspectratio}
% Set default figure placement to htbp
\makeatletter
\def\fps@figure{htbp}
\makeatother
\setlength{\emergencystretch}{3em} % prevent overfull lines
\providecommand{\tightlist}{%
  \setlength{\itemsep}{0pt}\setlength{\parskip}{0pt}}
\setcounter{secnumdepth}{5}
\usepackage{booktabs}
\usepackage{amsthm}
\makeatletter
\def\thm@space@setup{%
  \thm@preskip=8pt plus 2pt minus 4pt
  \thm@postskip=\thm@preskip
}
\makeatother
\usepackage[]{natbib}
\bibliographystyle{apalike}

\title{A First Course In Statistics}
\author{Andrea Mascaretti}
\date{2020-02-06}

\begin{document}
\maketitle

{
\setcounter{tocdepth}{1}
\tableofcontents
}
\hypertarget{prerequisites}{%
\chapter{Prerequisites}\label{prerequisites}}

Welcome to this introductory course in statistics with R!
This book will help you get started and will guide you through the material of the course.

The material of this course is based on the \textbf{moxier} package.
It consists of a series of \textbf{learnr} notebooks, \citep{R-learnr}, that will guide through your first steps into the wonderful world of statistics and statistical computing!

\hypertarget{getting-started}{%
\section{Getting started}\label{getting-started}}

Whilst we will dedicate Chapter \ref{intro} to the topic of software installation, we here introduce the main tools we will use throughout this course: \textbf{R} and \textbf{RStudio}.

\hypertarget{r}{%
\subsection{R}\label{r}}

So, what do we talk about when we talk about R? According to the \href{https://www.r-project.org/about.html}{R project} website, R is a software environment that includes:

\begin{quote}
\begin{itemize}
\tightlist
\item
  an effective data handling and storage facility,
\item
  a suite of operators for calculations on arrays, in particular matrices,
\item
  a large, coherent, integrated collection of intermediate tools for data analysis,
\item
  graphical facilities for data analysis and display either on-screen or on hardcopy, and
\item
  a well-developed, simple and effective programming language which includes conditionals, loops, user-defined recursive functions and input and output facilities.
\end{itemize}
\end{quote}

Quite a number of things! We will use this set of tools to dive deep into the principles of statistics and statistical computation.

Another thing that is worth noticing is that \textbf{R} is Free Software. It means anybody can contribute to its development. Common tools have emerged to solve problems. \textbf{R} can be extended with such tools, which are called \emph{packages} in R parlance, to do all sort of incredible things. Packages are usually stored on CRAN, the Comprehensive R Archive Network, and they range from packages to \href{https://rich-iannone.github.io/blastula/}{send emails} to \href{https://cran.r-project.org/web/views/MachineLearning.html}{machine learning}.

\hypertarget{rstudio}{%
\subsection{RStudio}\label{rstudio}}

\href{https://rstudio.com/}{\textbf{RStudio}} is an \textbf{R} IDE (Integrated Development Environment). But what does it mean? Basically, it is a set of software that helps you be more productive: it allows you to quickly manage files, see what variables you have defined and a vast number of other things. This book and the \textbf{moxier} package have been developed from within \textbf{RStudio}. We will see in a minute how to install it!

\hypertarget{some-useful-links}{%
\section{Some useful links}\label{some-useful-links}}

If you are interesting in learning more about \textbf{R} as a programming language, you can find many resources on the Internet.
Some nice books are \textbf{Hands-On Programming with R}, \citep{grolemund2014} to get started and \textbf{Advanced R} \citep{wickham2019} to dive deep into the features of the language.

\hypertarget{licence}{%
\section{Licence}\label{licence}}

The \textbf{moxier} package is subject to the \href{https://www.r-project.org/Licenses/GPL-3}{GPL-3} licence. For more information, visit \url{https://mascaretti.github.io/moxier/}.

This book is licensed under a Creative Commons Attribution-NonCommercial-ShareAlike 4.0 International License.

\hypertarget{intro}{%
\chapter{Introduction}\label{intro}}

We are now going to install the tools we need. We will start from \textbf{R}, to then move to \textbf{RStudio} and, finally, the packages we are going to need.

\hypertarget{installing-r}{%
\section{Installing R}\label{installing-r}}

Installing \textbf{R} has never been easier. You can simply follow the instructions you find in the video.

\includegraphics{https://player.vimeo.com/video/203516510}

\hypertarget{literature}{%
\chapter{Literature}\label{literature}}

Here is a review of existing methods.

\hypertarget{methods}{%
\chapter{Methods}\label{methods}}

We describe our methods in this chapter.

\hypertarget{applications}{%
\chapter{Applications}\label{applications}}

Some \emph{significant} applications are demonstrated in this chapter.

\hypertarget{example-one}{%
\section{Example one}\label{example-one}}

\hypertarget{example-two}{%
\section{Example two}\label{example-two}}

\hypertarget{final-words}{%
\chapter{Final Words}\label{final-words}}

We have finished a nice book.

  \bibliography{book.bib,packages.bib}

\end{document}
